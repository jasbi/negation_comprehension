\documentclass[man,floatsintext]{apa6}
\usepackage{lmodern}
\usepackage{amssymb,amsmath}
\usepackage{ifxetex,ifluatex}
\usepackage{fixltx2e} % provides \textsubscript
\ifnum 0\ifxetex 1\fi\ifluatex 1\fi=0 % if pdftex
  \usepackage[T1]{fontenc}
  \usepackage[utf8]{inputenc}
\else % if luatex or xelatex
  \ifxetex
    \usepackage{mathspec}
  \else
    \usepackage{fontspec}
  \fi
  \defaultfontfeatures{Ligatures=TeX,Scale=MatchLowercase}
\fi
% use upquote if available, for straight quotes in verbatim environments
\IfFileExists{upquote.sty}{\usepackage{upquote}}{}
% use microtype if available
\IfFileExists{microtype.sty}{%
\usepackage{microtype}
\UseMicrotypeSet[protrusion]{basicmath} % disable protrusion for tt fonts
}{}
\usepackage{hyperref}
\hypersetup{unicode=true,
            pdftitle={Comprehension of Negation in Toddlers},
            pdfauthor={~\&},
            pdfkeywords={keywords},
            pdfborder={0 0 0},
            breaklinks=true}
\urlstyle{same}  % don't use monospace font for urls
\usepackage{graphicx,grffile}
\makeatletter
\def\maxwidth{\ifdim\Gin@nat@width>\linewidth\linewidth\else\Gin@nat@width\fi}
\def\maxheight{\ifdim\Gin@nat@height>\textheight\textheight\else\Gin@nat@height\fi}
\makeatother
% Scale images if necessary, so that they will not overflow the page
% margins by default, and it is still possible to overwrite the defaults
% using explicit options in \includegraphics[width, height, ...]{}
\setkeys{Gin}{width=\maxwidth,height=\maxheight,keepaspectratio}
\IfFileExists{parskip.sty}{%
\usepackage{parskip}
}{% else
\setlength{\parindent}{0pt}
\setlength{\parskip}{6pt plus 2pt minus 1pt}
}
\setlength{\emergencystretch}{3em}  % prevent overfull lines
\providecommand{\tightlist}{%
  \setlength{\itemsep}{0pt}\setlength{\parskip}{0pt}}
\setcounter{secnumdepth}{0}
% Redefines (sub)paragraphs to behave more like sections
\ifx\paragraph\undefined\else
\let\oldparagraph\paragraph
\renewcommand{\paragraph}[1]{\oldparagraph{#1}\mbox{}}
\fi
\ifx\subparagraph\undefined\else
\let\oldsubparagraph\subparagraph
\renewcommand{\subparagraph}[1]{\oldsubparagraph{#1}\mbox{}}
\fi

%%% Use protect on footnotes to avoid problems with footnotes in titles
\let\rmarkdownfootnote\footnote%
\def\footnote{\protect\rmarkdownfootnote}


  \title{Comprehension of Negation in Toddlers}
    \author{\textsuperscript{1}~\& \textsuperscript{1}}
    \date{}
  
\shorttitle{Comprehension of Negation in Toddlers}
\affiliation{
\vspace{0.5cm}
\textsuperscript{1} \\\textsuperscript{2} }
\keywords{keywords\newline\indent Word count: X}
\usepackage{csquotes}
\usepackage{upgreek}
\captionsetup{font=singlespacing,justification=justified}

\usepackage{longtable}
\usepackage{lscape}
\usepackage{multirow}
\usepackage{tabularx}
\usepackage[flushleft]{threeparttable}
\usepackage{threeparttablex}

\newenvironment{lltable}{\begin{landscape}\begin{center}\begin{ThreePartTable}}{\end{ThreePartTable}\end{center}\end{landscape}}

\makeatletter
\newcommand\LastLTentrywidth{1em}
\newlength\longtablewidth
\setlength{\longtablewidth}{1in}
\newcommand{\getlongtablewidth}{\begingroup \ifcsname LT@\roman{LT@tables}\endcsname \global\longtablewidth=0pt \renewcommand{\LT@entry}[2]{\global\advance\longtablewidth by ##2\relax\gdef\LastLTentrywidth{##2}}\@nameuse{LT@\roman{LT@tables}} \fi \endgroup}


\usepackage{lineno}

\linenumbers

\authornote{Add complete departmental affiliations for each author here. Each new line herein must be indented, like this line.

Enter author note here.

Correspondence concerning this article should be addressed to , Postal address. E-mail: \href{mailto:my@email.com}{\nolinkurl{my@email.com}}}

\abstract{
Previous research suggests that children understand truth-conditional negation


}

\begin{document}
\maketitle

\hypertarget{introduction}{%
\section{Introduction}\label{introduction}}

A sign outside a taqueria reads: \enquote{\emph{everything either is or is not a taco}}. Apart from being true, and humerous, the sign is a good reminder of something deep about human language and thought. We are capable of forming and communicating abstract categories and concepts that go beyond what we experience. Negation is perhaps the best example of this capacity. It is an abstract concept that allows us to partition our conceptual space and bring under one umbrella, a variety of concepts that have nothing in common other than simply \enquote{not being something}; for example not beeing a taco! Most importantly, negation operates in a combinatorial and productive manner, allowing us to form negative concepts from any existing positive one. How does this capacity emerge in humans? What is the role of language? Does it have a single origin, or does it develop gradually through abstraction in multiple conceptual domains?

Previous research has proposed several possible origins for abstract and combinatorial negation including the hypothesized primitive concepts of \enquote{non-existence} and \enquote{rejection} (Bloom, 1970). In this paper, we consider and test a hypothesis that to our knowledge has not been explored before: that abstract combinatorial negation originates from word learning and the act of labeling, specifically labeling concrete objects with nominals. In short, negation allows the learner to know which entities are referred to by a label as well as which entities fall outside its extension. An important prediction of this hypothesis is that comprehension of negation emerges first in the context of labeling and word learning. In the next section, we provide a comprehensive review of previous experimental literature on the comprehension of negation, and show that the literature provides some initial support for this hypothesis. However, an important confound is that previous studies have used different methods and dependent measures in different linguistic contexts. As explained in Section \ref{current-study}, this paper tests toddler's comprehension of negation in different linguistic contexts within the same paradigm and using the same dependent measure. Experiment 1 tests todderls comprehension with predicative nominals in a the context of labeling. Experiment 2 tests their comprehension with locative prepositions, and Experiment 3 with simple transitives. Section ? presents our general discussion and conclusions.

\hypertarget{previous-research}{%
\subsection{Previous Research}\label{previous-research}}

We classify the literature on comprehension of negation along three dimensions: 1. which negative morpheme was investigated, 2. what type of argument the negative morpheme composed with, and 3. which experimental method was used to test children's comprehension. Comparison of studies along the first dimension reveals morpheme-specific development. Comparison along the second dimension shows how productive and compositional early negation is across constructions, and finally, the third dimension helps us see how successful different methods are in capturing children's competence with negation.

With respect to negative morphemes, previous work has tested children's comprehension of the English \emph{no} (Austin, Theakston, Lieven, \& Tomasello, 2014; Feiman, Mody, Sanborn, \& Carey, 2017; Nordmeyer \& Frank, 2014), \emph{not} (Austin et al., 2014; De Villiers \& Flusberg, 1975; Donaldson, 1970; Feiman et al., 2017; Kim, 1985; Pea, 1980, 1982; Slobin, 1966), and the clitic \emph{'nt} (Reuter, Feiman, \& Snedeker, 2018), as well as Korean \emph{ani} (Kim, 1985), German \emph{nein} (Hummer, Wimmer, \& Antes, 1993) and French \emph{ne \ldots{} pas} (de Carvalho, Barrault, \& Christophe, 2017). With respect to the type of argument, these negative morphemes composed with predicative nominals such as \enquote{(not) a dog} (de Carvalho et al., 2017; De Villiers \& Flusberg, 1975; Hummer et al., 1993; Kim, 1985; Pea, 1980, 1982) and \enquote{(no) apples} (Nordmeyer \& Frank, 2014), locative prepositional phrases such as \enquote{(not) in the bucket} (Austin et al., 2014; Feiman et al., 2017), or transitive present tense (Slobin, 1966) and past tense verb phrases (Reuter et al., 2018) such as \enquote{(not) chasing the dog} and \enquote{(didn't) break one of the plates}. Finally these studies used a variety of methods including picture verification with sentence completion, metalinguistic judgments, spontaneous verbal feedback, eye-tracking, and looking time as well as search behavior in a hide-and-seek paradigm and finally verbal elicitation tasks. Each of the three dimensions discussed above can affect the age at which children are shown to succeed in an experimental task. In what follows, we provide a brief summary of the results and conclusions of previous research, explaining how they fit this overall classificaiton. Figure \ref{fig:litreview} provides a graphic summary.

\begin{figure}
\centering
\includegraphics{negation_comprehension_files/figure-latex/litreview-1.pdf}
\caption{\label{fig:litreview}Graphic summary of the literature on comprehension of negation. The x-axis shows child age in years and months. The y-axis shows the negative morphemes and the type of linguistic constructions it combined with. Arrows show the age of success in each experiment. Boxes show where the experiments in this study fit the literature.}
\end{figure}

Early work on children's comprehension of negation was inspired by Wason's (1959, 1961) finding that in the absence of context, negative statements take longer to process than affirmative ones. Slobin (1966) tested comprehension of \emph{not} in 6, 8, 10, and 12-year-olds as well as adults in present tense transitive sentences such as \enquote{the cat is not chasing the dog}, using a picture verification task with metalinguistic judgments. Participants saw a picture, heard a positive or a negative sentence, and had to press one of two buttons to determine whether the sentence was \enquote{right} or \enquote{wrong}. The main dependent measure was how quickly participant responded. Confirming Wason's findings, the study reported the following order of response times: true-affirmatives \textless{} false-affirmatives \textless{} false-negatives \textless{} true-negatives. The study reported an error rate of 18.2\% for 6-year-olds but did not provide further explanations on error patterns.

Wason (1965) had also showed that the pragmatics of the task, specifically whether a negative statement is expected or plausible given the context facilitates processing of negative sentences. Donaldson (1970) tested 155 5-to-7-year-olds on a sentence completion and sentence comprehension task. In the sentence completion task, children saw 5 yellow circles in a row and a blue circle below them, and they had to complete a statment about the last circle such as \enquote{this circle is not \ldots{}}. In the sentence comprehension task, children saw a card with four shapes: black square, black circle, green square, green circle. The experimenter said she had a shape in mind and said the shape was \enquote{not a green square}. Children had to guess which shape the experimenter had in mind. Donaldson (1970) did not find Wason's pragmatic facliation with her sample but her results showed important task-based performance. In the sentence completion task, children were 100\% correct with positive sentences and only around 36\% correct with negative ones. With sentence comprehension, however, children's responses to simple negative statements were about 90\% correct.

De Villiers and Flusberg (1975) simplified the task in Donaldson (1970) by changing abstract shapes to everyday objects like a row of cars and single bottle, and tested 2-to-4-year-old children with plausible and implausible sentence completion. In plausible trials the experimenter pointed to the bottle and said \enquote{this is not a \ldots{}} (expected answer: car) while in implausible trials she pointed to one of the cars and said \enquote{this is not a \ldots{}} (expected answer: bottle). They found that all age groups responded to plausible negatives faster than implausible negatives. They also found relatively low rates of errors across age ranges (12.5\% in two-year-olds, 5\% in three-year-olds, and 4\% in four-year-olds), with the majority of them occurring with implausible negatives. The error rate is substantially lower than what is reported in Donaldson (1970) or Slobin (1966) for much older children. De Villiers and Flusberg (1975) provided evidence that even some two-year-olds could comprehend \emph{not} and complete negative sentences sucessfully.

Following Inhelder and Piaget (1958), Feldman (1972) was the first to bring up the importance of linguistic negation for abstract concept formation and tested children in controlled experiments. She reported five experiments on hunderds of children between the ages of 3 and 7 years. In these experiments, 18 wooden blocks with 3 shapes (star, circle, square) and 3 colors (red, green, black) were used with different forms of commands like \enquote{give me the things that are \emph{not} black}. The negative morpheme was always \emph{not} and it combined with adjectives like \emph{black} or plural nominals like \emph{squares} or \emph{circles}. Correct responses were those in which all objects that did not have the attribute were handed to the experimenter. The studies reported slow improvement in performance from age 3, with 6-7 year-olds reaching an asymptote with about 68\% correct responses in trials with negation, compared to 91\% correct in trials without negation. However, the study found that this is partly due to the large number of blocks that have to be handed to the experimenter. Performance improved when the number of blocks was reduced. Again, task-based confounds seemed to have underestimatd children's competence with negation.

Pea (1980, 1982) was also interested in the relation between language and logical thought. He used a picture verification task and tested 40 children, 10 in each age range of 18, 24, 30, and 36 months. Children were presented with a picture (e.g.~a ball) and the experimenter described the picture using a true affirmative (e.g. \enquote{that's the ball}), false affirmative (e.g. \enquote{that's the duck}), true negative (e.g. \enquote{that's not the duck}), or false negative statement (e.g. \enquote{that's not the ball}). The dependent measure was how often children corrected the experimenter by saying \enquote{no} or providing the right label. With false affirmatives, 24 and 36-month-olds provided the right lable more often. On the other hand, 30 and 36-moth-olds but not younger children provided more corrections by saying \enquote{no}. With false negatives, 24-, 30-, and 36-month-olds provided the correct lable more often but did not provide many corrections with \enquote{no}. Overall, the results suggested some success with the comprehension of negation in the 24-36 months age range.

Kim (1985) used a similar picture verification task, but instead participants had to provide explicit metalinguistic judgments. She tested English and Korean-speaking three-to-six-year-olds. Children saw pictures of common objects, heard a puppet describe them with an affirmative or a negative sentence (e.g. \enquote{\emph{this is not an apple}}), and had to judge whether the puppet was \enquote{right} or not (truth-value judgment task). The study found substantial proportions of errors (48\% overall) in children's judgments of true negatives (e.g. \enquote{\emph{this is not an apple}} when the picture was a banana). She concluded that the majority of children under five years of age had difficulty with negative sentences. Using slighly different methods, however, Hummer et al. (1993) reached the opposite conclusion. They tested one-to-three-year-old German-speaking children with similar linguistic stimuli, except that statements were turned into questions (e.g. \enquote{\emph{is this an apple?}}) and the dependent measure was children's yes/no responses. They found that between the ages of 2;4-2;7, 86\% of children's \enquote{no} resposnes were correct. Younger children provided the correct label, incorrectly said \enquote{yes}, or provided \enquote{other} types of answers and vocalisations. Overall, studies on children's development of logical language and thought showed the importance of task demands and the choice of dependent measure in estimates of children's success or failure with linguistic negation. Metalinguistic judgments (e.g. Kim, 1985) or tasks that involve many objects and several actions to be carried out (e.g. Feldman, 1972) may be less likely to show success with younger children. Tasks that rely on children's spontaneous reactions (e.g. Pea, 1980, 1982) or simple yes/no responses to questions (e.g. Hummer et al., 1993) may have a better chance of sucess with young children.

A few studies have used children's eye-movements as their dependent measures. Nordmeyer and Frank (2014) tested the comprehension of the determiner \emph{no} with plural nominals in commands like \enquote{look at the boy who has no apples}. Two- to four-year-olds and adults looked at items like the picture of two boys, one with no apples and one with two apples. Only four-year-olds and adults looked at the boy with no apples when they heard the command. In a second experiment, they tested participants with items like the picture of two boys, one with two apples and one with two gifts. This time three-year-olds also looked at the boy without the apples after hearing the negative command. In both experiments, the eye-movements of two-year-olds in positive and negative trials were similar, providing no evidence for comprehension of negation.

Reuter et al. (2018) tested children's comprehension of past tense auxiliary negator \emph{didn't}, composing with transitive verbs like \emph{break} and \emph{close} in sentence like \enquote{DW didn't break one of the plates} (experiment 1) or \enquote{show me the one DW didn't break} (experiment 2). Participants (two- and three-year-olds) saw pictures of two objects on the screen, one with the change of state due to the verb (e.g.~broken plate) and one without (e.g.~intact plate). The study found that three-year-olds looked at and pointed to the rigth object in affirmative and negative trials about 80\% of the times. Two-year-olds, however, were at chance unless affirmative trials preceded negative trials, in which case, they were also successful about 80\% of the times. Notice that both eye-tracking studies differed from other studies on almost all dimensions. They used negative morphemes (\emph{no} and \emph{didn't}) as well as compositional structures (object relative clause and transitive verbs) not tested before. They also differed from each other on these dimensions except for what they used as the dependent measure (i.e.~eye-gaze). Nevertheless with respect to the general issue of comprehension of negation, Reuter et al. (2018) provide success with the auxiliary negation \emph{didn't} and transitive verbs at three-years old as well as 2;5-2;9, with the caveat that children should first see the positive form.

Two studies have focused on early comprehension of negation with locative prepositions. Austin et al. (2014) tested 126 English-speaking children between the ages of 1;8 and 2;6. The dependent measure was children's searching behavior in two hiding locations: a house and a bucket. One experimenter hid a wooden block and another asked: \enquote{Is it in the \ldots{} (e.g.~bucket)?} The first experimenter responded using gustures (head nod vs.~head shake), single words (yes vs.~no), or verb phrase negation (e.g. \enquote{it's in the bucket} vs. \enquote{it's not in the bucket}). The study found that only the oldest age group (2;4-2;6) successfully used all forms of negation (gestural, single word, or sentential) to infer the correct location. The middle age group (2;0-2;2) succesfully interpreted single word and verb phrase negation but not gestural negation (head shakes). Surprisingly, children's performance was better with negative than positive statements even in the oldest age group. However, this surprising finding was not replicated by Feiman et al. (2017) who independently came up with the same paradigm as Austin et al. (2014). Feiman et al. (2017) found successful comprehension of affirmative locatives as well as negative locatives with \emph{no} and \emph{not} among children between 2;2-2;4. Younger children (1;7-2;1) were successful with affirmatives but not negatives. Most importantly, the youngest group (1;7-1;10) interpreted negatives like affirmatives, suggesting they ignored the negative morpheme on average. Finally, in their last experiment Feiman et al. (2017) showed the youngest group which container was empty and found that they had no problem finding the hidden object when they had visual information of absence. This provides evidence that the failure of the youngest group in the task is not due to task demands.

To summarize, previous research has provided evidence on what types of tasks have a better chance of capturing children's competence with negation. Specifically, tasks that do not rely on metalinguistic judgments, do not require children to attend to multiple objects or carry out several actions, and tasks that introduce the positive forms first fair better. The literature has also provided converging evidence that around two and a half years of age, children understand both negative response particles (i.e. \emph{no}, \emph{nein}) and verb phrase negation (i.e. \emph{not}) in combination with identity predicates and locative prepositions. Hummer et al. (1993) showed this for German \emph{nein} and identity predicates at 2;4-2;7. De Villiers and Flusberg (1975) showed success wtih \emph{not} and identity predicates in the same age range with a production task. Austin et al. (2014) and Feiman et al. (2017) provide evidence for English \emph{no} and \emph{not} in combination with locative prepositions at 2;3. Is it possible that negation in combination with identity predicates and locative prepositions are known earlier than our experiments have managed to capture so far?

Using a word learning paradigm, de Carvalho et al. (2017) provide evidence for comprehension with identity predicates in 18-month-olds. During habituation, toddlers saw a pinguin spinning and heard \enquote{Look, It's a bamoule!}. This would help them learn the novel noun \emph{bamoule} as \enquote{pinguin}. Then they saw the pinguin cartwheeling and heard \enquote{Look! She's pirdaling!}. This helped them learn the novel verb \emph{pirdale} as \enquote{cartwheeling}. During the test phase, they added negation to the sentences and also switched the videos. Half of the children saw the pinguin spinning and heard \enquote{Look! She is not pirdaling} (which is true if pridaling is learned as cartwheeling) and half saw the pinguin cartwheeling and heard \enquote{Look! It's not a bamoule!} (which is false if \emph{bamoule} is learned as pinguin). de Carvalho et al. (2017) found that 18-month-olds looked longer to the trials where negation was false than those in which negation was true. In the next section, we explain how our experiments build on de Carvalho et al. (2017) and extend it beyond labeling speech acts.

\hypertarget{current-study}{%
\subsection{Current Study}\label{current-study}}

We first start by conceptually replicating the findings of de Carvalho et al. (2017) in Experiment 1. In Experiment 2, we extend our paradigm to test comprehension of negation with locative prepositions. Experiment 3 tests toddler's comprehension of negation in transitive sentences. The goal is to see whether comprehension of negation is specific to labeling events or if it can generalize across compositional contexts.

\hypertarget{experiment-1}{%
\section{Experiment 1}\label{experiment-1}}

This experiment focuses on todders' ability to react to the truth altering property of negation with identity predicates. We habituated toddlers to positive and truth-preserving adverbs like \emph{really} as in \enquote{this is really a dog}. Then we tested them with either the similar truth-preserving adverb \emph{now} or the negative and truth-altering adverb \emph{not}. We predicted that if toddlers are sensitive to the functional role of negation in altering the truth conditions of statemets, they should dishabituate when they notice the change in truth conditions.

\hypertarget{methods}{%
\subsection{Methods}\label{methods}}

We conducted a looking time study that paired linguistic audio stimuli with images of objects appearing on a screen. The linguistic stimuli were simple sentences of the general form \enquote{this is {[}adverb{]} a {[}noun{]}} (e.g. \enquote{this is really a ball} vs. \enquote{this is not a ball}). The images were objects described by the nouns (e.g.~a ball). There were 6 adverbs (\emph{really},\emph{indeed},\emph{only},\emph{just},\emph{now}, and \emph{not}) and 3 nouns (\emph{ball}, \emph{dog}, and \emph{shoe}) resulting in 18 total possible trials. The dependent measure was how long toddlers looked at the screen. Figure \ref{fig:methodfig} shows the design of the experiment.

\begin{figure}
\centering
\includegraphics{negation_comprehension_files/figure-latex/methodfig-1.pdf}
\caption{\label{fig:methodfig}Study design, habituation, and test phases of the current study. Curly brackets represent within-participant randomized blocks while arrows represent between-participant randomization of trials.}
\end{figure}

\hypertarget{participants-materials}{%
\subsubsection{Participants \& Materials}\label{participants-materials}}

N toddlers in the age range of 18 to 30 months were tested. The experiment only used three images: a tennis ball, a dog, and a shoe (Figure). Images were selected from a free online repository. Using the MB-CDI data available through Wordbank (Frank, Braginsky, Yurovsky, \& Marchman, 2016), we made sure that half of toddlers in that age range produce the corresponding nouns (\emph{ball}, \emph{dog}, and \emph{shoe}) by 18 months and almost all toddlers produce them by age 24 months. 18 linguistic stimuli were recorded corresponding to the 18 combinations of adverbs and nouns. Recordings are available on the study's online repository.

\hypertarget{procedure}{%
\subsubsection{Procedure}\label{procedure}}

The study consisted of two phases: habituation and test. In habituation trials, toddlers heard positive sentences of the general form \enquote{this is {[}adverb{]} a {[}noun{]}}. Adverbs were randomly selected from the following set: \emph{indeed}, \emph{really}, \emph{just}, \emph{only}. After the adverb was selected, nouns were randomly selected from the following set: \emph{ball}, \emph{dog}, \emph{shoe}. Nouns were randomized within the the adverb blocks, and adverb blocks were in turn randomized as well. In each trial, first an attention getter appeard in the middle of the screen. When toddlers looked at the screen the object appeard and the audio of the sentence was played. The sentence was repeated three times. Trial ended when toddlers looked away for more than 2 seconds. All the sentences during the habituation phase were true of the pictures presented on the screen. In ther words, the adverbs did not alter the truth conditions of the statement. The habituation phase continued until toddler's mean looking time for three consecutive trials was reduced to 50\% of the mean looking time of their first three trials (Cohen and Gelber, 1975). After infants reached this criterion or were done with all 12 habituation trials, we started the test phase.

The test phase had two between-participant conditions: positive (control) and negative. The positive condition used the adverb \emph{now} and was similar to the habituation phase in that the adverb was positive and did not alter the truth conditions of the statement. The negative condition used the adverb \emph{not} which altered the truth conditions of the statement. Only in such negative trials, the statement was false with respect to the image on the screen. We predicted that if toddlers are sensitive to the truth conditional contributions of the adverb \emph{not}, they will dishabituate in the negative condition but not the positive (control) condition.

\hypertarget{results}{%
\subsection{Results}\label{results}}

We used R (Version 3.6.0; R Core Team, 2019) and the R-packages \emph{jpeg} (Version 0.1.8; Urbanek, 2014), \emph{papaja} (Version 0.1.0.9842; Aust \& Barth, 2018), and \emph{png} (Version 0.1.7; Urbanek, 2013) for all our analyses.

\hypertarget{discussion}{%
\subsection{Discussion}\label{discussion}}

\hypertarget{experiment-2}{%
\section{Experiment 2}\label{experiment-2}}

\hypertarget{experiment-3}{%
\section{Experiment 3}\label{experiment-3}}

\hypertarget{general-discussion}{%
\section{General Discussion}\label{general-discussion}}

\hypertarget{appendix}{%
\section{Appendix}\label{appendix}}

\newpage

\hypertarget{references}{%
\section{References}\label{references}}

\begingroup
\setlength{\parindent}{-0.5in}
\setlength{\leftskip}{0.5in}

\hypertarget{refs}{}
\leavevmode\hypertarget{ref-R-papaja}{}%
Aust, F., \& Barth, M. (2018). \emph{papaja: Create APA manuscripts with R Markdown}. Retrieved from \url{https://github.com/crsh/papaja}

\leavevmode\hypertarget{ref-austin2014denial}{}%
Austin, K., Theakston, A., Lieven, E., \& Tomasello, M. (2014). Young children's understanding of denial. \emph{Developmental Psychology}, \emph{50}(8), 2061.

\leavevmode\hypertarget{ref-bloom1970language}{}%
Bloom, L. (1970). \emph{Language development: Form and function in emerging grammars}. The MIT Press.

\leavevmode\hypertarget{ref-deCarvalhoEtal2017}{}%
de Carvalho, A., Barrault, A., \& Christophe, A. (2017). ``Look! It is not a bamoule!'' 18-month-olds understand negative sentences. Boston, MA: 42nd Boston University Conference on Language Development - BUCLD.

\leavevmode\hypertarget{ref-deVilliersFlusberg1975}{}%
De Villiers, J. G., \& Flusberg, H. B. T. (1975). Some facts one simply cannot deny. \emph{Journal of Child Language}, \emph{2}(2), 279--286.

\leavevmode\hypertarget{ref-donaldson1970developmental}{}%
Donaldson, M. (1970). Developmental aspects of performance with negatives, gbf d'Arcais, wjm levelt, editors, advances in psycholinguistics. North-Holland, Amsterdam-London.

\leavevmode\hypertarget{ref-feiman2017nonot}{}%
Feiman, R., Mody, S., Sanborn, S., \& Carey, S. (2017). What do you mean, no? Toddlers' comprehension of logical ``no'' and ``not''. \emph{Language Learning and Development}, \emph{13}(4), 430--450.

\leavevmode\hypertarget{ref-feldman1972}{}%
Feldman, S. (1972). Children's understanding of negation as a logical operation. \emph{Genetic Psychology Monographs}, \emph{85}, 3--49.

\leavevmode\hypertarget{ref-frank2016wordbank}{}%
Frank, M. C., Braginsky, M., Yurovsky, D., \& Marchman, V. A. (2016). Wordbank: An open repository for developmental vocabulary data. \emph{Journal of Child Language}, \emph{44}(3), 677--694.

\leavevmode\hypertarget{ref-hummer1993origins}{}%
Hummer, P., Wimmer, H., \& Antes, G. (1993). On the origins of denial negation. \emph{Journal of Child Language}, \emph{20}(3), 607--618.

\leavevmode\hypertarget{ref-inhelder1958growth}{}%
Inhelder, B., \& Piaget, J. (1958). \emph{The growth of logical thinking from childhood to adolescence: An essay on the construction of formal operational structures}. Basic Books Publishers.

\leavevmode\hypertarget{ref-kim1985development}{}%
Kim, K. (1985). Development of the concept of truth-functional negation. \emph{Developmental Psychology}, \emph{21}(3), 462.

\leavevmode\hypertarget{ref-nordmeyer2014negation}{}%
Nordmeyer, A. E., \& Frank, M. C. (2014). The role of context in young children's comprehension of negation. \emph{Journal of Memory and Language}, \emph{77}, 25--39.

\leavevmode\hypertarget{ref-pea1980logic}{}%
Pea, R. D. (1980). Logic in early child language. \emph{Annals of the New York Academy of Sciences}.

\leavevmode\hypertarget{ref-pea1982origins}{}%
Pea, R. D. (1982). Origins of verbal logic: Spontaneous denials by two-and three-year olds. \emph{Journal of Child Language}, \emph{9}(3), 597--626.

\leavevmode\hypertarget{ref-R-base}{}%
R Core Team. (2019). \emph{R: A language and environment for statistical computing}. Vienna, Austria: R Foundation for Statistical Computing. Retrieved from \url{https://www.R-project.org/}

\leavevmode\hypertarget{ref-reuter2018no}{}%
Reuter, T., Feiman, R., \& Snedeker, J. (2018). Getting to no: Pragmatic and semantic factors in two-and three-year-olds' understanding of negation. \emph{Child Development}, \emph{89}(4), e364--e381.

\leavevmode\hypertarget{ref-slobin1966grammatical}{}%
Slobin, D. I. (1966). Grammatical transformations and sentence comprehension in childhood and adulthood. \emph{Journal of Verbal Learning and Verbal Behavior}, \emph{5}(3), 219--227.

\leavevmode\hypertarget{ref-R-png}{}%
Urbanek, S. (2013). \emph{Png: Read and write png images}. Retrieved from \url{https://CRAN.R-project.org/package=png}

\leavevmode\hypertarget{ref-R-jpeg}{}%
Urbanek, S. (2014). \emph{Jpeg: Read and write jpeg images}. Retrieved from \url{https://CRAN.R-project.org/package=jpeg}

\leavevmode\hypertarget{ref-wason1959processing}{}%
Wason, P. C. (1959). The processing of positive and negative information. \emph{Quarterly Journal of Experimental Psychology}, \emph{11}(2), 92--107.

\leavevmode\hypertarget{ref-wason1961response}{}%
Wason, P. C. (1961). Response to affirmative and negative binary statements. \emph{British Journal of Psychology}, \emph{52}(2), 133--142.

\leavevmode\hypertarget{ref-wason1965contexts}{}%
Wason, P. C. (1965). The contexts of plausible denial. \emph{Journal of Verbal Learning and Verbal Behavior}, \emph{4}(1), 7--11.

\endgroup


\end{document}
