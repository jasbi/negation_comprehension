\documentclass[man,floatsintext]{apa6}
\usepackage{lmodern}
\usepackage{amssymb,amsmath}
\usepackage{ifxetex,ifluatex}
\usepackage{fixltx2e} % provides \textsubscript
\ifnum 0\ifxetex 1\fi\ifluatex 1\fi=0 % if pdftex
  \usepackage[T1]{fontenc}
  \usepackage[utf8]{inputenc}
\else % if luatex or xelatex
  \ifxetex
    \usepackage{mathspec}
  \else
    \usepackage{fontspec}
  \fi
  \defaultfontfeatures{Ligatures=TeX,Scale=MatchLowercase}
\fi
% use upquote if available, for straight quotes in verbatim environments
\IfFileExists{upquote.sty}{\usepackage{upquote}}{}
% use microtype if available
\IfFileExists{microtype.sty}{%
\usepackage{microtype}
\UseMicrotypeSet[protrusion]{basicmath} % disable protrusion for tt fonts
}{}
\usepackage{hyperref}
\hypersetup{unicode=true,
            pdftitle={Comprehension of Sentential Negation in Toddlers},
            pdfauthor={First Author~\& Ernst-August Doelle},
            pdfkeywords={keywords},
            pdfborder={0 0 0},
            breaklinks=true}
\urlstyle{same}  % don't use monospace font for urls
\usepackage{graphicx,grffile}
\makeatletter
\def\maxwidth{\ifdim\Gin@nat@width>\linewidth\linewidth\else\Gin@nat@width\fi}
\def\maxheight{\ifdim\Gin@nat@height>\textheight\textheight\else\Gin@nat@height\fi}
\makeatother
% Scale images if necessary, so that they will not overflow the page
% margins by default, and it is still possible to overwrite the defaults
% using explicit options in \includegraphics[width, height, ...]{}
\setkeys{Gin}{width=\maxwidth,height=\maxheight,keepaspectratio}
\IfFileExists{parskip.sty}{%
\usepackage{parskip}
}{% else
\setlength{\parindent}{0pt}
\setlength{\parskip}{6pt plus 2pt minus 1pt}
}
\setlength{\emergencystretch}{3em}  % prevent overfull lines
\providecommand{\tightlist}{%
  \setlength{\itemsep}{0pt}\setlength{\parskip}{0pt}}
\setcounter{secnumdepth}{0}
% Redefines (sub)paragraphs to behave more like sections
\ifx\paragraph\undefined\else
\let\oldparagraph\paragraph
\renewcommand{\paragraph}[1]{\oldparagraph{#1}\mbox{}}
\fi
\ifx\subparagraph\undefined\else
\let\oldsubparagraph\subparagraph
\renewcommand{\subparagraph}[1]{\oldsubparagraph{#1}\mbox{}}
\fi

%%% Use protect on footnotes to avoid problems with footnotes in titles
\let\rmarkdownfootnote\footnote%
\def\footnote{\protect\rmarkdownfootnote}


  \title{Comprehension of Sentential Negation in Toddlers}
    \author{First Author\textsuperscript{1}~\& Ernst-August Doelle\textsuperscript{1,2}}
    \date{}
  
\shorttitle{Comprehension of Negation in Toddlers}
\affiliation{
\vspace{0.5cm}
\textsuperscript{1} Wilhelm-Wundt-University\\\textsuperscript{2} Konstanz Business School}
\keywords{keywords\newline\indent Word count: X}
\usepackage{csquotes}
\usepackage{upgreek}
\captionsetup{font=singlespacing,justification=justified}

\usepackage{longtable}
\usepackage{lscape}
\usepackage{multirow}
\usepackage{tabularx}
\usepackage[flushleft]{threeparttable}
\usepackage{threeparttablex}

\newenvironment{lltable}{\begin{landscape}\begin{center}\begin{ThreePartTable}}{\end{ThreePartTable}\end{center}\end{landscape}}

\makeatletter
\newcommand\LastLTentrywidth{1em}
\newlength\longtablewidth
\setlength{\longtablewidth}{1in}
\newcommand{\getlongtablewidth}{\begingroup \ifcsname LT@\roman{LT@tables}\endcsname \global\longtablewidth=0pt \renewcommand{\LT@entry}[2]{\global\advance\longtablewidth by ##2\relax\gdef\LastLTentrywidth{##2}}\@nameuse{LT@\roman{LT@tables}} \fi \endgroup}


\usepackage{lineno}

\linenumbers

\authornote{Add complete departmental affiliations for each author here. Each new line herein must be indented, like this line.

Enter author note here.

Correspondence concerning this article should be addressed to First Author, Postal address. E-mail: \href{mailto:my@email.com}{\nolinkurl{my@email.com}}}

\abstract{
Previous research suggests that children understand truth-conditional negation


}

\begin{document}
\maketitle

\hypertarget{introduction}{%
\section{Introduction}\label{introduction}}

\hypertarget{previous-research}{%
\subsection{Previous Research}\label{previous-research}}

Austin, Theakston, Lieven, \& Tomasello (2014)

Feiman, Mody, Sanborn, and Carey (2017)

de Carvalho, Barrault, and Christophe (2017) pinguin cartwheeling: the essence of their finding is what we test in this study.

Hungarian study

\hypertarget{current-study}{%
\section{Current Study}\label{current-study}}

\hypertarget{methods}{%
\section{Methods}\label{methods}}

We conducted a looking time study that paired linguistic audio stimuli with images of objects appearing on a screen. The linguistic stimuli were simple sentences of the general form \enquote{this is {[}adverb{]} a {[}noun{]}} (e.g. \enquote{this is really a ball} vs. \enquote{this is not a ball}). The images were objects described by the noun (e.g.~a ball). There were 6 adverbs (\emph{really},\emph{indeed},\emph{only},\emph{just},\emph{now}, and \emph{not}) and 3 nouns (\emph{ball}, \emph{dog}, and \emph{shoe}) resulting in 18 total trials. The dependent measure was how long toddlers looked at the screen. Figure \ref{fig:methodfig} shows the design of the experiment.

\begin{figure}
\centering
\includegraphics{negation_comprehension_files/figure-latex/methodfig-1.pdf}
\caption{\label{fig:methodfig}Study design, habituation, and test phases of the current study. Curley brackets represent within-participant randomized blocks while arrows represent between-participant randomization of trials.}
\end{figure}

\hypertarget{participants-materials}{%
\subsection{Participants \& Materials}\label{participants-materials}}

N toddlers in the age range of 18 to 30 months were tested. The experiment only used three images: a tennis ball, a dog, and a shoe (Figure). Images were selected from a free online repository. Using the MB-CDI data available through Wordbank (Frank, Braginsky, Yurovsky, \& Marchman, 2016), we made sure that half of toddlers in that age range produce the corresponding nouns (\emph{ball}, \emph{dog}, and \emph{shoe}) by 18 months and almost all toddlers produce them by age 24 months. 18 linguistic stimuli were recorded corresponding to the 18 combinations of adverbs and nouns. Recordings are available on the study's online repository.

\hypertarget{procedure}{%
\subsection{Procedure}\label{procedure}}

The study consisted of two phases: habituation and test. In habituation trials, toddlers heard positive sentences of the general form \enquote{this is {[}adverb{]} a {[}noun{]}}. Adverbs were randomly selected from the following set: \emph{indeed}, \emph{really}, \emph{just}, \emph{only}. After the adverb was selected, nouns were randomly selected from the following set: \emph{ball}, \emph{dog}, \emph{shoe}. Nouns were randomized within the the adverb blocks, and adverb blocks were in turn randomized as well. In each trial, first an attention getter appeard in the middle of the screen. When toddlers looked at the screen the object appeard and the audio of the sentence was played. The sentence was repeated three times. Trial ended when toddlers looked away for more than 2 seconds. All the sentences during the habituation phase were true of the pictures presented on the screen. In ther words, the adverbs did not alter the truth conditions of the statement. The habituation phase continued until toddler's mean looking time for three consecutive trials was reduced to 50\% of the mean looking time of their first three trials (Cohen and Gelber, 1975). After infants reached this criterion or were done with all 12 habituation trials, we started the test phase.

The test phase had two between-participant conditions: positive (control) and negative. The positive condition used the adverb \emph{now} and was similar to the habituation phase in that the adverb was positive and did not alter the truth conditions of the statement. The negative condition used the adverb \emph{not} which altered the truth conditions of the statement. Only in such negative trials, the statement was false with respect to the image on the screen. We predicted that if toddlers are sensitive to the truth conditional contributions of the adverb \emph{not}, they will dishabituate in the negative condition but not the positive (control) condition.

\hypertarget{results}{%
\section{Results}\label{results}}

We used R (Version 3.6.0; R Core Team, 2019) and the R-packages \emph{jpeg} (Version 0.1.8; Urbanek, 2014), and \emph{papaja} (Version 0.1.0.9842; Aust \& Barth, 2018) for all our analyses.

\hypertarget{discussion}{%
\section{Discussion}\label{discussion}}

\hypertarget{appendix}{%
\section{Appendix}\label{appendix}}

\newpage

\hypertarget{references}{%
\section{References}\label{references}}

\begingroup
\setlength{\parindent}{-0.5in}
\setlength{\leftskip}{0.5in}

\hypertarget{refs}{}
\leavevmode\hypertarget{ref-R-papaja}{}%
Aust, F., \& Barth, M. (2018). \emph{papaja: Create APA manuscripts with R Markdown}. Retrieved from \url{https://github.com/crsh/papaja}

\leavevmode\hypertarget{ref-frank2016wordbank}{}%
Frank, M. C., Braginsky, M., Yurovsky, D., \& Marchman, V. A. (2016). Wordbank: An open repository for developmental vocabulary data. \emph{Journal of Child Language}, \emph{44}(3), 677--694.

\leavevmode\hypertarget{ref-R-base}{}%
R Core Team. (2019). \emph{R: A language and environment for statistical computing}. Vienna, Austria: R Foundation for Statistical Computing. Retrieved from \url{https://www.R-project.org/}

\leavevmode\hypertarget{ref-R-jpeg}{}%
Urbanek, S. (2014). \emph{Jpeg: Read and write jpeg images}. Retrieved from \url{https://CRAN.R-project.org/package=jpeg}

\endgroup


\end{document}
